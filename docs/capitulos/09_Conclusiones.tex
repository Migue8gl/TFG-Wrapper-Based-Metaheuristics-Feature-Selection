\chapter{Conclusiones}
En este trabajo, se ha abordado el desafiante problema de la selección de características, fundamental en el ámbito del aprendizaje automático y la minería de datos. Este proceso se revela como crucial no solo para mejorar la eficiencia y precisión de los modelos predictivos, si no también para facilitar su interpretación y reducir la complejidad computacional asociada. A lo largo de esta investigación, se ha explorado el uso de metaheurísticas como herramientas efectivas para optimizar este proceso en entornos tanto binarios como continuos.\\[6pt]
La motivación detrás de este estudio ha sido proporcionar una evaluación exhaustiva y objetiva de diversas metaheurísticas utilizadas en la selección de características. Se ha investigado cómo estas técnicas se comportan en términos de precisión, reducción, tiempo de ejecución e incluso variabilidad, utilizando conjuntos de datos de referencia y metodologías de validación cruzada para asegurar la robustez de los resultados.\\[6pt]
Además, se ha analizado la transferibilidad de estas técnicas entre dominios binarios y continuos, explorando las ventajas y limitaciones de cada enfoque. Este análisis ha permitido identificar las fortalezas y debilidades de cada metaheurística en función del tipo de representación de las características, proporcionando así conocimiento sobre su rendimiento relativo y sus aplicaciones más adecuadas.\\[6pt]
Las metaheurísticas binarias han sido retratadas como la versión algorítmica más útil aplicada a la selección de características. No obstante, codificaciones reales son también capaces de abordar este problema.\\[6pt]
Metaheuristicas modernos tales como \textbf{GWO} o \textbf{CS} se han mostrado como algoritmos punteros en el problema de selección de características, demostrando su flexibilidad en multitud de problemas y en distintas versiones de codificación. Los algoritmos clásicos también han sido probados, comparados y analizados, demostrando una vez más su robustez y sus buenos resultados a pesar de la multitud de propuestas que han ido apareciendo a lo largo de los años que podrían haberlos opacado.\\[6pt]
A nivel personal, este proyecto ha supuesto un desafío a muchos niveles. Nunca antes había abordado una planificación tan grande, tampoco había realizado un estudio previo tan extenso en el tiempo y tan exhaustivo. Por todo ello, la realización de este trabajo ha supuesto para mí un antes y un después en mi vida profesional y académica, dado que me ha brindado habilidades y conocimientos tanto prácticos como teóricos. Este trabajo me ha enseñado a planificar un proyecto a largo plazo y de gran escala, me ha empujado a indagar, investigar e informarme en fuentes confiables y a plasmar esa información recogida en código, me ha educado en técnicas de programación, buenas prácticas y a trabajar en un proyecto de investigación, me ha enseñado a realizar comparaciones y análisis rigurosos. Sobre todo me ha preparado en ser constante, tener paciencia y buscar la perfección en lo que hago.\\[6pt]
Gracias a estos años de carrera en la facultad de \textit{Ingeniería Informática} he podido cursar gran cantidad de asignaturas que me han ayudado a realizar este proyecto. Asignaturas como \textbf{PDOO} (Programación y Diseño Orientado a Objetos), que me han formado en los paradigmas de diseño de software y abstracción de conceptos, \textbf{MH} (Metaheuristicas), que me han preparado en el entendimiento y aplicación de técnicas avanzadas de optimización y búsqueda heurística, \textbf{AA} que me ha proporcionado los fundamentos teóricos y prácticos necesarios para entender los algoritmos de aprendizaje automático y su aplicación en problemas de análisis de datos, \textbf{PTC} (Programación Técnica y Científica), que han desarrollado mis conocimientos y habilidades en el lenguaje de programación \textbf{Python}, así como me ha instruido en los distintos procesos que han de llevarse a cabo en una tarea de investigación en el ámbito del aprendizaje automático, y por último, \textbf{Prácticas de Empresa}, que me ha dado entre muchas otras cosas, una visión real de mis capacidades en el mundo de la informática.