% Desenvolvido por: Prof. Dr. David Buzatto
% Modificado por: Brian Sena. 
% Versão 0.1
% Data: 01/08/2023
%\documentclass[notes, aspectratio=169]{beamer}
\documentclass[aspectratio=169]{beamer}
\usepackage{structure}
\usepackage{array}
\usepackage{graphicx}
\usepackage{makecell}
\usepackage{animate}
\usepackage{changepage}
\renewcommand\theadalign{bc}
\renewcommand\theadfont{\bfseries}
\renewcommand\theadgape{\Gape[4pt]}
\renewcommand\cellgape{\Gape[4pt]}
 
\newcommand\blfootnote[1]{%
  \begingroup
  \renewcommand\thefootnote{}\footnote{#1}%
  \addtocounter{footnote}{-1}%
  \endgroup
}


\captionsetup{justification=centering}


%\usepackage[alf,abnt-emphasize=bf]{abntex2cite}
%\usepackage[backend=biber, defernumbers=true, style=numeric-comp, bibstyle=ieee, sorting=none]{biblatex}
\usepackage[style=verbose,maxnames=10,babel=hyphen,hyperref=true,abbreviate=true,backend=biber,mcite]{biblatex}
% Configurando BibLaTeX
\DefineBibliographyStrings{spanish}{
  url = {URL},
  andothers={et ~al\adddot}
}

\addbibresource{bibliografia/bibliografia.bib}

\setbeamercovered{transparent}

\setbeamertemplate{endpage}{%
  \subsection*{Dudas, preguntas o comentarios.}
  \begin{frame}[noframenumbering]
      \frametitle{Agradecimientos}
      \begin{center}
      \Huge Gracias por su atención.
      \par\bigskip
      \large ¿Dudas, preguntas o comentarios?
      \end{center}
    \end{frame}
}

\AtEndDocument{\usebeamertemplate{endpage}}
\begin{document}

\titulo{Estudio y Análisis de Metaheurísticas Modernos para la Selección de Características}
\autor{Miguel García López}
\orientador{Daniel Molina Cabrera}
\github{\url{https://github.com/Migue8gl/TFG-Wrapper-Based-Metaheuristics-Feature-Selection}}

\title[TFG]{\titulo}
\author[Miguel García López]{Miguel García López}
\institute[UGR]{Universidad de Granada}
\date{\today}

\curso{Grado en Ingeniería Informática}

\local{Granada}

\chapter*{}
%\thispagestyle{empty}
%\cleardoublepage

%\thispagestyle{empty}

\cleardoublepage
\thispagestyle{empty}

\begin{center}
       {\large\bfseries Estudio y Análisis de Metaheurísticas modernas para el problema de Selección de Características}\\
\end{center}
\begin{center}
       MIGUEL GARCÍA LÓPEZ\\
\end{center}

%\vspace{0.7cm}
\noindent{\textbf{Palabras clave}: Metaheurística, selección de características, binario, optimización, población, fitness, aprendizaje automático}\\

\vspace{0.7cm}
\noindent{\textbf{Resumen}}\\

En el ámbito del \textit{Machine Learning}, algunos algoritmos, como KNN o SVM, muestran excelentes resultados, pero a menudo requieren un procesamiento previo para identificar las características más relevantes. Esto se debe a que, ya sea por la amplitud y complejidad del problema, por falta de conocimiento o contexto al recolectar los datos, es posible aglomerar información innecesaria, dando lugar a conjuntos de datos inmensos. Incluso los algoritmos que internamente realizan esta identificación pueden beneficiarse de un procesamiento adicional. Este preprocesamiento, conocido como selección de características (\textit{feature selection}), se considera un problema complejo de optimización combinatoria.\\[6pt]

Existen varios métodos comunes para la selección de características: filtrado, envoltura (Wrapper), métodos basados en árboles, Análisis de Componentes Principales (PCA) y selección L1 (Lasso Regression).\\[6pt]

Estos métodos pueden utilizarse individualmente o en combinación para mejorar la selección de características y el rendimiento de los modelos de \textit{Machine Learning}. En este documento se hablarán de métodos metaheurísticos para la resolución de este problema.\\[6pt]

Las metaheurísticas son algoritmos diseñados para resolver problemas de optimización complejos cuando los recursos son limitados. Aunque inicialmente se desarrollaron para abordar principalmente problemas combinatorios, en la actualidad se están proponiendo y aplicando cada vez más en problemas que implican variables continuas o reales. En respuesta a la creciente demanda en el campo de la selección de características, se han adaptado versiones especializadas de estas metaheurísticas para abordar este tipo de problemas combinatorios. Sin embargo, a pesar del número creciente de propuestas en este campo, las comparaciones objetivas entre ellas son limitadas. Aunque existen revisiones bibliográficas, muchas de ellas carecen de comparaciones adecuadas debido a la importancia y actualidad del problema de selección de características. Por lo tanto, hay una necesidad de estudios que proporcionen una evaluación comparativa más rigurosa y exhaustiva de las diferentes propuestas en este ámbito.\\[6pt]

Existe por tanto una necesidad de estudios que proporcionen una evaluación comparativa más rigurosa y exhaustiva de las diferentes propuestas en este ámbito, planteando un doble estudio que use tanto adaptaciones binarias como reales con algoritmos modernos.\\[6pt]

En este trabajo de carácter científico, se llevará a cabo una revisión bibliográfica de diversas metaheurísticas recientes para abordar el problema de selección de características. Se estudiarán e implementarán aquellas consideradas más prometedoras, con el objetivo de construir un repertorio amplio y variado de propuestas. Posteriormente, se realizará un estudio comparativo exhaustivo utilizando diversos algoritmos de machine learning y conjuntos de datos representativos. Finalmente, se llevará a cabo un análisis crítico utilizando diversas métricas y valoraciones, como la tasa de acierto y el tiempo de ejecución, entre otras.
\cleardoublepage


\thispagestyle{empty}


\begin{center}
       {\large\bfseries Study and Analysis of Modern Metaheuristics for the Feature Selection Problem}\\
\end{center}
\begin{center}
       MIGUEL GARCÍA LÓPEZ\\
\end{center}

%\vspace{0.7cm}
\noindent{\textbf{Keywords}: Metaheuristic, feature selection, binary, optimization, population, fitness, machine learning}\\

\vspace{0.7cm}
\noindent{\textbf{Abstract}}\\

In the field of Machine Learning, some algorithms, such as KNN or SVM, show excellent results, but often require preprocessing to identify the most relevant features. This is because, whether due to the broad and complex nature of the problem, lack of knowledge, or lack of context when collecting data, it is possible to aggregate unnecessary information, resulting in immense datasets. Even algorithms that internally perform this identification can benefit from additional processing. This preprocessing, known as feature selection, is considered a complex combinatorial optimization problem.

There are several common methods for feature selection: filtering, wrapping (Wrapper), tree-based methods, Principal Component Analysis (PCA), and L1 selection (Lasso Regression).\\[6pt]

These methods can be used individually or in combination to improve feature selection and the performance of \textit{Machine Learning} models. This document will discuss metaheuristic methods for solving this problem.\\[6pt]

Metaheuristics are algorithms designed to solve complex optimization problems when resources are limited. Initially developed mainly for combinatorial problems, they are increasingly proposed and applied to problems involving continuous or real variables. In response to the growing demand in feature selection, specialized versions of these metaheuristics have been adapted to tackle combinatorial problems. However, despite the increasing number of proposals in this field, objective comparisons between them are limited. Although there are literature reviews, many lack adequate comparisons due to the importance and timeliness of the feature selection problem. Therefore, there is a need for studies providing a more rigorous and exhaustive comparative evaluation of different proposals in this area.\\[6pt]

Therefore, there is a need for studies that provide a more rigorous and comprehensive comparative evaluation of the different proposals in this field, proposing a dual study that uses both binary and real adaptations with modern algorithms.\\[6pt]

In this scientific work, a literature review of various recent metaheuristics for feature selection will be conducted. The most promising ones will be studied and implemented to build a broad and varied repertoire of proposals. Subsequently, a comprehensive comparative study will be conducted using various machine learning algorithms and representative datasets. Finally, a critical analysis will be carried out using various metrics and assessments, such as accuracy rate and execution time, among others.\\[6pt]

\chapter*{}
\thispagestyle{empty}

\noindent\rule[-1ex]{\textwidth}{2pt}\\[4.5ex]

Yo, \textbf{Miguel García López}, alumno de la titulación Ingeniería Informática de la \textbf{Escuela Técnica Superior
       de Ingenierías Informática y de Telecomunicación de la Universidad de Granada}, con DNI 77159865E, autorizo la
ubicación de la siguiente copia de mi Trabajo Fin de Grado en la biblioteca del centro para que pueda ser
consultada por las personas que lo deseen.

\vspace{6cm}

\noindent Fdo: Miguel García López

\vspace{2cm}

\begin{flushright}
       Granada a 29 de Enero de 2024.
\end{flushright}


\chapter*{}
\thispagestyle{empty}

\noindent\rule[-1ex]{\textwidth}{2pt}\\[4.5ex]

D. \textbf{Daniel Molina Cabrera}, Profesor del Departamento Ciencias de la Computación e Inteligencia Artificial de la Universidad de Granada.

\vspace{0.5cm}
\textbf{Informan:}

\vspace{0.5cm}

Que el presente trabajo, titulado \textit{\textbf{Estudio y Análisis de Metaheurísticas modernas para el problema de Selección de Características}},
ha sido realizado bajo su supervisión por \textbf{Miguel García López}, y autorizamos la defensa de dicho trabajo ante el tribunal
que corresponda.

\vspace{0.5cm}

Y para que conste, expiden y firman el presente informe en Granada a 29 de Enero de 2024.

\vspace{1cm}

\textbf{El director:}

\vspace{5cm}

\noindent \textbf{Daniel Molina Cabrera}

\chapter*{Agradecimientos}
\thispagestyle{empty}

\vspace{1cm}


Me gustaría agradecer a mi tutor Daniel Molina Cabrera, profesor del Departamento de Ciencias de la Computación e Inteligencia Artificial de la Universidad de Granada, por su apoyo, instrucción y paciencia en la realización de este proyecto. Gracias a él y a la gran cantidad de tiempo que ha invertido en mí, este proyecto ha sido posible.\\[6pt] 
Segundo, quería agradecer a mi pareja, por su constante apoyo, por su atención y escucha en todo momento, incluso cuando no entendía mis explicaciones efusivas y confusas, por la motivación que me brindaba. Por todo ello le doy gracias.\\[6pt]
Gracias al lector, por dedicar su tiempo a la lectura de este documento, que con tanto cariño y empeño he redactado.\\[6pt]
Y por último gracias, a todas las personas que me rodean y me hacen la vida un poquito más bonita.\\[6pt]

\section{Introducción}

\subsection{Contexto}
\begin{frame}
  \frametitle{Introducción a la Selección de Características}
  \begin{columns}
    \column{0.5\textwidth}
    \begin{enumerate}
      \item \textbf{La selección de características} es un proceso crucial en el aprendizaje automático
            \begin{itemize}
              \item Implica elegir un subconjunto de características relevantes.
              \item Es un problema \textbf{NP duro}
            \end{itemize}
            \item{Importancia de la reducción de dimensionalidad}:
            \begin{itemize}
              \item Mejora la generalización y precisión del modelo
              \item Reduce el ruido en los datos
            \end{itemize}
    \end{enumerate}
    \column{0.5\textwidth}
    \begin{figure}
      \begin{center}
        \includegraphics[width=\textwidth]{imagenes/chapter1/feature_selection_pros-removebg-preview.png}
      \end{center}
      \caption{Beneficios de la selección de características}
    \end{figure}
  \end{columns}
  \vspace{-.2cm}
\end{frame}

\note{

}

\subsection{Motivación}
\begin{frame}
  \frametitle{Motivación}
  \begin{enumerate}
    \item \textbf{La selección de características} es un procesamiento esencial en el aprendizaje automático
    \item Muchos algoritmos y métodos abordar el problema. Nos centramos en
          metaheurísticas
    \item No hay gran número de comparaciones entre versiones binarias y continuas
    \item Poca rigurosidad en comparaciones existentes
  \end{enumerate}
  \vspace{-.2cm}
\end{frame}

\begin{frame}
  \frametitle{Motivación}
  \begin{enumerate}
    \item Identificamos propuestas novedosas (problemas continuos)
    \item Implementamos versiones binarias de las anteriores
    \item Comparación rigurosas entre algoritmos:
          \begin{itemize}
            \item Distintas métricas
            \item Test estadísticos
          \end{itemize}
  \end{enumerate}
  \vspace{-.2cm}
\end{frame}

\subsection{Búsquedas Scopus}
\begin{frame}
  \frametitle{Tendencia Scopus}
  \begin{columns}
    \column{0.5\textwidth}
    \begin{figure}
      \begin{center}
        \includegraphics[width=\textwidth]{imagenes/chapter3/scopus_chart.png}
      \end{center}
      \caption{Tendencia en artículos publicados de \textbf{feature selection} en Scopus. Se incrementa exponencialmente con el tiempo.}
    \end{figure}

    \column{0.5\textwidth}
    \begin{figure}
      \begin{center}
        \includegraphics[width=\textwidth]{imagenes/chapter3/scopus_chart2.png}
      \end{center}
      \caption{Tendencia en artículos publicados sobre \textbf{feature selection} usando \textbf{metaheurísticas} en Scopus.}
    \end{figure}
  \end{columns}
\end{frame}

\note{

}

\note{

}

\subsection{Objetivos}
\begin{frame}
  \frametitle{Objetivos}
  \begin{columns}
    \column{0.6\textwidth}
    \begin{enumerate}
      \item Evaluar el desempeño de las metaheurísticas
      \item Investigar versiones continuas y binarias
      \item Fortalezas y debilidades de las metaheurísticas
      \item Recomendaciones prácticas según el contexto
      \item Evaluación de resultados finales y realizar comparaciones
      \item Resolver una serie de preguntas de investigación. ¿Binarios vs. continuos?
    \end{enumerate}
    \column{0.6\textwidth}
    \begin{figure}
      \begin{center}
        \includegraphics[width=0.7\textwidth]{imagenes/chapter1/real_vs_bin.png}
      \end{center}
      \caption{Codificaciones binarias y continuas\footnotemark[1]}
    \end{figure}
  \end{columns}
  \footnotetext[1]{\cite{binary_real}}
\end{frame}

\note{

}

\subsection{Metaheurísticas}
\begin{frame}
  \frametitle{¿Qué son las metaheurísticas?}
  \begin{columns}
    \column{0.6\textwidth}
    \begin{enumerate}
      \item Las metaheurísticas son algoritmos de \textbf{optimización}
      \item Suelen estar bioinspiradas
      \item Son algoritmos generales que no dependen de un dominio en un problema específico
      \item Consiguen soluciones muy buenas en tiempos razonables
    \end{enumerate}
    \column{0.4\textwidth}
    \begin{figure}
      \begin{center}
        \animategraphics[loop, autoplay, width=0.8\textwidth]{10}{imagenes/chapter1/ParticleSwarmArrowsAnimation-}{0}{99}
      \end{center}
      \caption{Algoritmo PSO en busca del \textbf{mínimo} de una función\footnotemark[2]}
    \end{figure}
  \end{columns}
  \footnotetext[2]{\cite{wiki:PSO}}
\end{frame}

\note{

}

\begin{frame}
  \frametitle{Tipos de mateheurísticas}
  \begin{columns}
    \column{0.4\textwidth}
    \begin{figure}
      \begin{center}
        \includegraphics[width=1\textwidth]{imagenes/chapter1/mh_euler_graph.png}
      \end{center}
      \caption{Tipos de metaheurísticas}
    \end{figure}
    \column{0.6\textwidth}
    \begin{enumerate}
      \item Las metaheurísticas seleccionadas en este proyecto son todas del tipo \textbf{poblacional} y evolutivas
    \end{enumerate}
  \end{columns}
\end{frame}

\note{

}


\note{

}
\section{Planificación}
\subsection{Presupuesto y planificación}

\begin{frame}
    \frametitle{Tareas}
    \begin{table}[htp]
        \centering
        \begin{tabular}{c|c|c}
            Tarea                         & Duración Prevista (h) & Duración Final (h) \\ \hline
            Investigación inicial         & 20                    & 20                 \\
            Diseño del software           & 10                    & 10                 \\
            Investigación metaheurísticas & 60                    & 60                 \\
            Implementación del software   & 70                    & 90                 \\
            Pruebas y refactorizado       & 15                    & 25                 \\
            Análisis de resultados        & 40                    & 20                 \\
            Documentación                 & 90                    & 120                \\ \hline
        \end{tabular}
        \caption{Tabla de duración de cada tarea}
        \label{tab:task_duration}
    \end{table}
\end{frame}

\begin{frame}
    \frametitle{Diagrama de Gantt}
    \begin{figure}
        \centering
        \includegraphics[width=0.8\textwidth]{imagenes/chapter2/gantt-fin.png}
        \caption{Diagrama de Gantt}
    \end{figure}
\end{frame}

\begin{frame}
    \frametitle{Presupuesto}
    \begin{columns}
        \column{1\textwidth}
        \begin{itemize}
            \item \textbf{Sueldo}: $25$€/hora, $340$ horas, total $8.500$€.
            \item \textbf{Portátil}: HP Pavilion, $1000$€, usado $8$ meses, costo $133.3$€.
            \item \textbf{Servidor}: Uso del servidor \textbf{Hércules}, costo real cero. Estimación en \textit{Google Cloud}.
        \end{itemize}
    \end{columns}
\end{frame}

\note{

}

\begin{frame}
    \frametitle{Presupuesto}
    \begin{columns}
        \column{1\textwidth}
        \begin{table}[htp]
            \centering
            \begin{tabular}{|l|r|}
                \hline
                \textbf{Item}                    & \textbf{Costo (€)} \\ \hline
                Salario                          & 8500               \\
                Ordenador portátil               & 133.3              \\
                Servidor CPU - GC Compute Engine & 75.84              \\
                \textbf{Total}                   & \textbf{8709.14}   \\ \hline
            \end{tabular}
            \caption{Costo estimado del proyecto}
        \end{table}
    \end{columns}
\end{frame}
\section{Revisión de la literatura}

\iffalse
  \begin{frame}
    \frametitle{Ventajas actuales de las Metaheurísticas}
    \begin{columns}
      \column{0.5\textwidth}
      \begin{figure}
        \begin{center}
          \includegraphics[width=\textwidth]{imagenes/chapter3/pbt.png}
        \end{center}
        \caption{Uso de algoritmo poblacional \textbf{PBT} en búsqueda de hiperparámetros \footnotemark[3]}
      \end{figure}
      \column{0.5\textwidth}
      \begin{itemize}
        \item \textbf{Implementación práctica}: Utilizadas en aplicaciones reales, como la planificación de rutas, diseño de redes y gestión de recursos
        \item \textbf{Integración con AI}: Combinadas con técnicas de inteligencia artificial para mejorar el rendimiento
        \item \textbf{Adaptabilidad}: Ajustables a problemas específicos mediante parametrización de los cuales no es posible obtener funciones de pérdida diferenciables
      \end{itemize}
    \end{columns}
    \footnotetext[3]{\cite{Mafarja201825}}
  \end{frame}

\fi 

\subsection{Algoritmos}
\begin{frame}
  \frametitle{Algoritmos seleccionados}
  \begin{enumerate}
    \item Se escogen para el proyecto una serie de algoritmos basándose en la investigación de aquellos más novedosos, con mejor rendimiento y más citados. Estos son los que se denominarán \textbf{modernos}
    \item Además de los algoritmos más novedosos, se incluyen una serie de algoritmos clásicos, cuyo robustez a lo largo de los años tras multitud de aplicaciones en problemas es notable. Esta categoría, es la de los algoritmos \textbf{clásicos}
  \end{enumerate}
\end{frame}

\begin{frame}
  \frametitle{Grey Wolf Optimizer}
  \begin{columns}
    \column{0.5\textwidth}
    \begin{enumerate}
      \item Inspirado en el comportamiento social y la técnica de caza de los lobos grises
      \item Modela una \textbf{jerarquía social} para guiar la búsqueda de soluciones óptimas, donde los lobos alfa, beta y delta lideran el proceso de exploración y explotación
    \end{enumerate}
    \column{0.5\textwidth}
    \begin{figure}
      \begin{center}
        \includegraphics[width=\textwidth]{imagenes/chapter3/grey-wolf-hunt.png}
      \end{center}
      \caption{Caza de los lobos grises \footnotemark[4]}
    \end{figure}
  \end{columns}
  \footnotetext[4]{\cite{mirjalili_grey_2014}}
\end{frame}

\begin{frame}
  \frametitle{Grasshopper Optimization Algorithm}
  \begin{columns}
    \column{0.5\textwidth}
    \begin{figure}
      \begin{center}
        \includegraphics[width=\textwidth]{imagenes/chapter3/goa-position-convergence.png}
      \end{center}
      \caption{Convergencia de los saltamontes \footnotemark[5]}
    \end{figure}
    \column{0.5\textwidth}
    \begin{enumerate}
      \item Simula el movimiento y la interacción de los saltamontes en sus distintas etapas de vida
    \end{enumerate}
  \end{columns}
  \footnotetext[5]{\cite{saremi_grasshopper_2017}}
\end{frame}

\begin{frame}
  \frametitle{Firefly Algorithm}
  \begin{columns}
    \column{0.5\textwidth}
    \begin{enumerate}
      \item Inspirado en el comportamiento de \textbf{parpadeo} y \textbf{atracción} de las luciérnagas
      \item Utiliza la intensidad de la luz como guía para la atracción entre luciérnagas, donde las luciérnagas menos brillantes se mueven hacia las más brillantes
    \end{enumerate}
    \column{0.5\textwidth}
    \begin{figure}
      \begin{center}
        \includegraphics[width=\textwidth]{imagenes/chapter3/firefly.jpg}
      \end{center}
      \caption{Imagen de una libélula con su característico brillo}
    \end{figure}
  \end{columns}
\end{frame}

\begin{frame}
  \frametitle{Cuckoo Search}
  \begin{columns}
    \column{0.5\textwidth}
    \begin{figure}
      \begin{center}
        \includegraphics[width=0.6\textwidth]{imagenes/chapter3/cucko_eggs.jpg}
      \end{center}
      \caption{Cuatro nidos de huevos de pájaro. En cada uno de ellos un huevo visiblemente más grande del pájaro Cuco \footnotemark[6]}
    \end{figure}
    \column{0.5\textwidth}
    \begin{enumerate}
      \item Inspirado en el comportamiento de anidación de los cucos y el \textbf{parasitismo} de puesta
      \item Caracterizado por usar métodos de búsqueda aleatoria y el método \textbf{Levy flight} para la exploración del espacio de soluciones
    \end{enumerate}
  \end{columns}
  \footnotetext[6]{\cite{chiswickchap_cuckooeggs_2024}}
\end{frame}

\begin{frame}
  \frametitle{Genetic Algorithm}
  \begin{columns}
    \column{0.5\textwidth}
    \begin{enumerate}
      \item Algoritmo basado en la recombinación de cromosomas, que toma inspiración de la \textbf{evolución} biológica y genética
      \item Hace uso de operadores tales como la \textbf{mutación} o el \textbf{cruce}
    \end{enumerate}
    \column{0.5\textwidth}
    \begin{figure}
      \begin{center}
        \includegraphics[width=\textwidth]{imagenes/chapter3/ga-working-principle.png}
      \end{center}
      \caption{Principio básico del algoritmo GA \footnotemark[7]}
    \end{figure}
  \end{columns}
  \footnotetext[7]{\cite{mathew2012genetic}}
\end{frame}

\begin{frame}
  \frametitle{Whale Optimization Algorithm}
  \begin{columns}
    \column{0.5\textwidth}
    \begin{figure}
      \begin{center}
        \includegraphics[width=0.7\textwidth]{imagenes/chapter3/spiral-update-position-wao.png}
      \end{center}
      \caption{Espiral para simular el mecanismo de ataque de la red de burbujas de las ballenas jorobadas \footnotemark[8]}
    \end{figure}
    \column{0.5\textwidth}
    \begin{enumerate}
      \item Inspirado en el comportamiento de las ballenas jorobadas
      \item Usa principalmente dos variantes de operadores de caza:
            \begin{itemize}
              \item \textbf{Espiral de búsqueda}
              \item \textbf{Técnica de burbujeo de red}
            \end{itemize}
    \end{enumerate}
  \end{columns}
  \footnotetext[8]{\cite{mirjalili_whale_2016}}
\end{frame}

\begin{frame}
  \frametitle{Artificial Bee Colony Optimization}
  \begin{columns}
    \column{0.5\textwidth}
    \begin{enumerate}
      \item Simula la \textbf{búsqueda de alimentos} de las abejas empleadas, las abejas observadoras y las abejas exploradoras para encontrar soluciones óptimas
    \end{enumerate}
    \column{0.5\textwidth}
    \begin{figure}
      \begin{center}
        \includegraphics[width=0.5\textwidth]{imagenes/chapter3/abco.png}
      \end{center}
      \caption{Diagrama de funcionamiento del ABCO \footnotemark[9]}
    \end{figure}
  \end{columns}
  \footnotetext[9]{\cite{Karaboga2009108}}
\end{frame}


\begin{frame}
  \frametitle{Dragonfly Algorithm}
  \begin{columns}
    \column{0.5\textwidth}
    \begin{figure}
      \begin{center}
        \includegraphics[width=0.7\textwidth]{imagenes/chapter3/da-operators.png}
      \end{center}
      \caption{Operadores del algoritmo DA \footnotemark[10]}
    \end{figure}
    \column{0.5\textwidth}
    \begin{enumerate}
      \item Basado en el comportamiento de enjambre y formación de las libélulas, usando operadores que controlan características como la \textbf{cohesión} de grupo o \textbf{distanciamiento} del enemigo, entre otros
      \item Simula las interacciones sociales y el movimiento de las libélulas para equilibrar la exploración y explotación del espacio de soluciones
    \end{enumerate}
  \end{columns}
  \footnotetext[10]{\cite{Meraihi2020}}
\end{frame}

\begin{frame}
  \frametitle{Ant Colony Optimization}
  \begin{columns}
    \column{0.5\textwidth}
    \begin{enumerate}
      \item Simula las colonias de hormigas. Para ello usa el rastro de \textbf{feromonas} para guiar la búsqueda de soluciones óptimas, donde las hormigas depositan y siguen feromonas en los caminos más prometedores
    \end{enumerate}
    \column{0.5\textwidth}
    \begin{figure}
      \begin{center}
        \includegraphics[width=0.5\textwidth]{imagenes/chapter3/aco.png}
      \end{center}
      \caption{Caminos de un grafo marcados por la feromona, operador esencial de ACO}
    \end{figure}
  \end{columns}
\end{frame}

\begin{frame}
  \frametitle{Particle Swarm Optimization}
  \begin{columns}
    \column{0.5\textwidth}
    \begin{figure}
      \begin{center}
        \includegraphics[width=0.7\textwidth]{imagenes/chapter3/pso.png}
      \end{center}
      \caption{Partículas en el espacio (con una velocidad y dirección) convergiendo en la iteración $N$}
    \end{figure}
    \column{0.5\textwidth}
    \begin{enumerate}
      \item Inspirado en el comportamiento social de los enjambres de aves y peces
      \item Simula la búsqueda colectiva de soluciones, donde cada partícula ajusta su posición basada en su \textbf{experiencia} personal y la de sus \textbf{vecinos}
    \end{enumerate}
  \end{columns}
\end{frame}

\begin{frame}
  \frametitle{Bat Algorithm}
  \begin{columns}
    \column{0.5\textwidth}
    \begin{enumerate}
      \item Basado en el comportamiento de \textbf{ecolocalización} de los murciélagos
      \item Utiliza la técnica de emisión de pulsos y el ajuste de frecuencia para explorar y explotar el espacio de soluciones
    \end{enumerate}
    \column{0.5\textwidth}
    \begin{figure}
      \begin{center}
        \includegraphics[width=0.9\textwidth]{imagenes/chapter3/ba.png}
      \end{center}
      \caption{Funcionamiento del algoritmo BA}
    \end{figure}
  \end{columns}
\end{frame}

\begin{frame}
  \frametitle{Differential Evolution}
  \begin{columns}
    \column{0.5\textwidth}
    \begin{figure}
      \begin{center}
        \includegraphics[width=1\textwidth]{imagenes/chapter3/de-crossover.png}
      \end{center}
      \caption{Operador de \textit{crossover} o cruce de DE \footnotemark[11]}
    \end{figure}
    \column{0.5\textwidth}
    \begin{enumerate}
      \item Utiliza la combinación y mutación de vectores solución para buscar la mejor solución, enfocándose en la \textbf{diferencia} entre las soluciones actuales para generar nuevas
    \end{enumerate}
  \end{columns}
  \footnotetext[11]{\cite{10.5555/1557464}}
\end{frame}
\section{Diseño Experimental}
\subsection{Conjuntos de datos}
\begin{frame}
  \frametitle{Conjuntos de datos}
  \begin{columns}
    \column{0.5\textwidth}
    \begin{enumerate}
      \item Se escogen conjuntos de datos por:
            \begin{itemize}
              \item \textbf{Variedad de áreas}
              \item \textbf{Diversidad de problemas}
              \item \textbf{Número de características}
              \item \textbf{Relevancia práctica}
            \end{itemize}
    \end{enumerate}
    \column{0.5\textwidth}
    \begin{table}[htp]
      \centering
      \tiny
      \setlength{\tabcolsep}{4pt}
      \begin{tabular}{ l r r r l }
        \hline
        \textbf{Dataset} & \textbf{Inst.} & \textbf{Car.} & \textbf{Cls.} & \textbf{Área} \\ \hline
        sonar            & 207            & 60            & 2             & Biología      \\
        spambase-460     & 459            & 54            & 2             & Informática   \\
        spectf-heart     & 348            & 44            & 2             & Medicina      \\
        waveform5000     & 5000           & 40            & 3             & Física        \\
        ionosphere       & 350            & 34            & 2             & Meteorología  \\
        dermatology      & 366            & 34            & 6             & Medicina      \\
        wdbc             & 568            & 29            & 2             & Medicina      \\
        parkinsons       & 200            & 22            & 2             & Medicina      \\
        zoo              & 101            & 18            & 7             & Biología      \\
        wine             & 182            & 13            & 3             & Alimentación  \\
        breast-cancer    & 286            & 9             & 2             & Medicina      \\
        diabetes         & 768            & 8             & 2             & Medicina      \\
        yeast            & 1483           & 8             & 10            & Biología      \\
        ecoli            & 336            & 7             & 8             & Biología      \\
        iris             & 149            & 4             & 3             & Biología      \\ \hline
      \end{tabular}
      \caption{Información de conjuntos de datos por número de características}
      \label{tab:datasets_info}
    \end{table}
  \end{columns}
\end{frame}

\begin{frame}
  \frametitle{Diseño Experimental}
  \begin{enumerate}
    \item La función \textit{fitness} se construye con:
          \begin{itemize}
            \item \textbf{Accuracy} al $90\%$.
            \item \textbf{Reducción} al $10\%$.
          \end{itemize}
    \item Se define como: \begin{equation}
            fitness = acc\cdot\alpha + red\cdot(1-\alpha)
            \label{eq:fitness}
          \end{equation}
          Donde $\alpha$ es la ponderación dada a la precisión o \textit{accuracy}.
  \end{enumerate}
\end{frame}
\section{Resultados y Análisis}

\begin{frame}
    \frametitle{Comparativas}
    \begin{enumerate}
        \item Se muestran resultados de los experimentos
        \item Hay muchas más comparativas en la sección del apéndice en la memoria principal:
        \begin{itemize}
            \item Figuras de convergencia
            \item Figuras de \textit{boxplots}
            \item Tablas comparativas en todas las métricas (\textit{fitness}, \textit{accuracy}, reducción, tiempo de ejecución, etc.)
            \item Tablas con resultados de los test estadísticos
        \end{itemize}
    \end{enumerate}
\end{frame}

\subsection{Preguntas de Investigación}
\begin{frame}
    \frametitle{Preguntas de Investigación}
    \begin{itemize}
        \item<1-> ¿Merece la pena el uso de algoritmos específicos para la selección de características?
        \item<2-> ¿Cómo se comparan los algoritmos recientes con los clásicos?
        \item<3-> ¿Cuáles de los algoritmos recientes parecen más prometedores?
        \item<4-> ¿Son los algoritmos igualmente eficaces en su versión binaria y continua?
        \item<5-> ¿Cuáles son las opciones más interesantes en ciertos contextos?
    \end{itemize}
\end{frame}

\subsection{Continuos}

\begin{frame}
    \frametitle{Ranking en continuos para fitness: kNN}
    \begin{figure}
        \begin{center}
            \includegraphics[width=0.45\textwidth]{imagenes/chapter5/rankings_knn_avg_real.png}
        \end{center}
        \caption{Ranking de los algoritmos en versión continua para kNN}
    \end{figure}
\end{frame}
\begin{frame}
    \frametitle{Ranking en continuos para fitness: SVC}
    \begin{figure}
        \begin{center}
            \includegraphics[width=0.45\textwidth]{imagenes/chapter5/rankings_svc_avg_real.png}
        \end{center}
        \caption{Ranking de los algoritmos en versión continua para SVC}
    \end{figure}
\end{frame}

\begin{frame}
    \frametitle{Resultados en continuos}
    \begin{itemize}
        \item<1-> Los mejores algoritmos en \textit{fitness} son:
            \begin{itemize}
                \item<1-> \textcolor{green}{\textbf{CS (Cuckoo Search)}}
                \item<1-> \textcolor{green}{\textbf{GWO (Grey Wolf Optimizer)}}
            \end{itemize}
        \item<2-> Los peores algoritmos son:
            \begin{itemize}
                \item<2-> \textcolor{red}{\textbf{GOA (Grasshopper Optimization Algorithm)}}
                \item<2-> \textcolor{red}{\textbf{DA (Dragonfly Algorithm)}}
            \end{itemize}
        \item<3-> Mejor rendimiento en reducción de características:
            \begin{itemize}
                \item<3-> \textcolor{blue}{\textbf{CS y GWO}} (con mucha diferencia)
            \end{itemize}
        \item<4-> Eficiencia temporal:
            \begin{itemize}
                \item<4-> Más rápido: \textcolor{orange}{\textbf{FA (Firefly Algorithm)}}
                \item<4-> Más lento: \textcolor{purple}{\textbf{ABCO (Artificial Bee Colony Optimization)}}
            \end{itemize}
    \end{itemize}
\end{frame}


\begin{frame}
    \frametitle{Convergencia en continuo: Ionosphere - kNN}
    \begin{figure}[htp]
        \includegraphics[width=0.45\textwidth]{imagenes/chapter5/optimizers_fitness_knn_io.png}
        \caption{Convergencia de todas las metaheurísticas en ionosphere - knn - real}
    \end{figure}
\end{frame}
\begin{frame}
    \frametitle{Convergencia en continuo: Diabetes - kNN}
    \begin{figure}[htp]
        \includegraphics[width=0.45\textwidth]{imagenes/chapter5/optimizers_fitness_knn_dia.png}
        \caption{Convergencia de todas las metaheurísticas en diabetes - knn - real}
    \end{figure}
\end{frame}


\subsection{Binarios}
\begin{frame}
    \frametitle{Ranking en binario para fitness: kNN}
    \begin{figure}
        \begin{center}
            \includegraphics[width=0.45\textwidth]{imagenes/chapter5/rankings_knn_avg_bin.png}
        \end{center}
        \caption{Ranking de los algoritmos en versión binaria para kNN}
    \end{figure}
\end{frame}
\begin{frame}
    \frametitle{Ranking en binario para fitness: SVC}
    \begin{figure}
        \begin{center}
            \includegraphics[width=0.45\textwidth]{imagenes/chapter5/rankings_svc_avg_bin.png}
        \end{center}
        \caption{Ranking de los algoritmos en versión binaria para SVC}
    \end{figure}
\end{frame}

\begin{frame}
    \frametitle{Resultados en binarios}
    \begin{itemize}
        \item<1-> Los mejores algoritmos en \textit{fitness} son:
            \begin{itemize}
                \item<1-> \textcolor{green}{\textbf{bGWO (Binary Grey Wolf Optimizer)}}
                \item<1-> \textcolor{green}{\textbf{bPSO (Binary Particle Swarm Optimization)}}
            \end{itemize}
        \item<2-> Los peores algoritmos son:
            \begin{itemize}
                \item<2-> \textcolor{red}{\textbf{bGOA (Binary Grasshopper Optimization Algorithm)}}
                \item<2-> \textcolor{red}{\textbf{bABCO (Binary Artificial Bee Colony Optimization)}}
            \end{itemize}
        \item<3-> Mejor rendimiento en reducción de características:
            \begin{itemize}
                \item<3-> \textcolor{blue}{\textbf{ACO (Ant Colony Optimization)}}
            \end{itemize}
        \item<4-> Eficiencia temporal:
            \begin{itemize}
                \item<4-> Más rápido: \textcolor{orange}{\textbf{bFA (Binary Firefly Algorithm)}}
                \item<4-> Más lento: \textcolor{purple}{\textbf{bABCO (Binary Artificial Bee Colony Optimization)}}
            \end{itemize}
    \end{itemize}
\end{frame}

\subsection{Comparación de resultados}
\begin{frame}
    \frametitle{Comparación de algoritmos específicos vs. originales}
    \begin{columns}
        \column{0.6\textwidth}
        \includegraphics[width=\textwidth]{imagenes/chapter6/selected_rate_comparison.png}
        \column{0.4\textwidth}
        \begin{itemize}
            \item<1-> Los algoritmos binarios tienden a seleccionar menos características
            \item<2-> Algoritmos continuos también capaces de reducir características
            \item<3-> \textcolor{blue}{Conclusión:} Ambos enfoques son válidos para la selección de características, pero los binarios son más eficaces
        \end{itemize}
    \end{columns}
\end{frame}

\begin{frame}
    \frametitle{Comparación de rendimiento: Recientes vs. Clásicos}
    \begin{columns}
        \column{0.6\textwidth}
        \includegraphics[width=\textwidth]{imagenes/chapter6/avg_comparison.png}
        \column{0.4\textwidth}
        \begin{itemize}
            \item<1-> Algoritmos recientes (GWO, CS) muestran excelente rendimiento
            \item<2-> Algunos clásicos (PSO, DE, GA) siguen siendo competitivos
            \item<3-> \textcolor{blue}{Conclusión:} Algunos recientes ofrecen mejoras, pero los clásicos siguen siendo muy robustos
        \end{itemize}
    \end{columns}
\end{frame}

\begin{frame}
    \frametitle{Algoritmos recientes prometedores}
    \begin{itemize}
        \item<1-> En versiones continuas:
            \begin{itemize}
                \item \textcolor{green}{\textbf{GWO (Grey Wolf Optimizer)}}
                \item \textcolor{green}{\textbf{CS (Cuckoo Search)}}
            \end{itemize}
        \item<2-> En versiones binarias:
            \begin{itemize}
                \item \textcolor{blue}{\textbf{bGWO (Binary Grey Wolf Optimizer)}}
                \item \textcolor{blue}{\textbf{bPSO (Binary Particle Swarm Optimization)}}
            \end{itemize}
        \item<3-> \textcolor{orange}{Conclusión:} GWO destaca en ambas versiones, mostrando gran adaptabilidad y rendimiento
    \end{itemize}
\end{frame}

\begin{frame}
    \frametitle{Eficacia en versiones originales vs. binarias}
    \begin{columns}
        \column{0.6\textwidth}
        \includegraphics[width=\textwidth]{imagenes/chapter6/avg_comparison.png}
        \column{0.4\textwidth}
        \begin{itemize}
            \item<1-> Rendimiento generalmente similar
            \item<2-> Ligeras ventajas para versiones continuas en algunos casos
            \item<3-> \textcolor{blue}{Conclusión:} La mayoría de algoritmos mantienen su eficacia al ser adaptados a versiones binarias
        \end{itemize}
    \end{columns}
\end{frame}

\begin{frame}
    \frametitle{Opciones más interesantes por contexto}
    \begin{itemize}
        \item<1-> Mejor fitness general:
            \begin{itemize}
                \item Continuo: \textcolor{green}{\textbf{CS, GWO}}
                \item Binario: \textcolor{blue}{\textbf{bGWO, bPSO}}
            \end{itemize}
        \item<2-> Mejor reducción de características:
            \begin{itemize}
                \item Continuo: \textcolor{green}{\textbf{CS, GWO}} (con mucha diferencia)
                \item Binario: \textcolor{blue}{\textbf{ACO}}
            \end{itemize}
        \item<3-> Mayor eficiencia temporal:
            \begin{itemize}
                \item Continuo: \textcolor{orange}{\textbf{FA (Firefly Algorithm)}}
                \item Binario: \textcolor{orange}{\textbf{bFA (Binary Firefly Algorithm)}}
            \end{itemize}
        \item<4-> \textcolor{purple}{Conclusión:} La elección del algoritmo depende de las prioridades específicas del problema (fitness, reducción de características, eficiencia)
    \end{itemize}
\end{frame}
\section{Conclusiones}
\subsection{Respuestas}

\begin{frame}
    \frametitle{Preguntas investigadas}
    \begin{columns}
        \column{0.5\textwidth}
        \begin{figure}
            \begin{center}
                \includegraphics[width=1\textwidth]{imagenes/chapter6/selected_rate_comparison.png}
            \end{center}
            \caption{Comparación binaria vs continuo en selección de características}
        \end{figure}
        \column{0.5\textwidth}
        \begin{itemize}
            \item ¿Merece pues la pena el uso de algoritmos específicos para la selección de características o las versiones originales son totalmente capaces de reducir?
        \end{itemize}
    \end{columns}
\end{frame}

\begin{frame}
    \frametitle{Preguntas investigadas}
    \begin{figure}
        \begin{center}
            \includegraphics[width=0.6\textwidth]{imagenes/chapter6/avg_comparison.png}
        \end{center}
        \caption{Comparación de \textit{fitness} en binario vs continuo}
    \end{figure}
\end{frame}

\begin{frame}
    \frametitle{Preguntas investigadas}
    \begin{columns}
        \column{0.5\textwidth}
        \begin{itemize}
            \item ¿Cómo son los recientes en comparación con los más clásicos?
            \item ¿Cuáles de los recientes parecen más prometedores?
        \end{itemize}
        \column{0.5\textwidth}
        \begin{itemize}
            \item ¿Son los algoritmos buenos en su versión original igualmente eficaces en su versión binaria?
            \item ¿Cuáles son las opciones más interesantes dentro de ciertos contextos?
        \end{itemize}
    \end{columns}
\end{frame}

\end{document} 

