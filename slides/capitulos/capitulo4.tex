\section{Diseño Experimental}
\subsection{Conjuntos de datos}
\begin{frame}
  \frametitle{Conjuntos de datos}
  \begin{columns}
    \column{0.5\textwidth}
    \begin{enumerate}
      \item Se escogen conjuntos de datos por:
            \begin{itemize}
              \item \textbf{Variedad de áreas}
              \item \textbf{Diversidad de problemas}
              \item \textbf{Número de características}
              \item \textbf{Relevancia práctica}
            \end{itemize}
    \end{enumerate}
    \column{0.5\textwidth}
    \begin{table}[htp]
      \centering
      \tiny
      \setlength{\tabcolsep}{4pt}
      \begin{tabular}{ l r r r l }
        \hline
        \textbf{Dataset} & \textbf{Inst.} & \textbf{Car.} & \textbf{Cls.} & \textbf{Área} \\ \hline
        sonar            & 207            & 60            & 2             & Biología      \\
        spambase-460     & 459            & 54            & 2             & Informática   \\
        spectf-heart     & 348            & 44            & 2             & Medicina      \\
        waveform5000     & 5000           & 40            & 3             & Física        \\
        ionosphere       & 350            & 34            & 2             & Meteorología  \\
        dermatology      & 366            & 34            & 6             & Medicina      \\
        wdbc             & 568            & 29            & 2             & Medicina      \\
        parkinsons       & 200            & 22            & 2             & Medicina      \\
        zoo              & 101            & 18            & 7             & Biología      \\
        wine             & 182            & 13            & 3             & Alimentación  \\
        breast-cancer    & 286            & 9             & 2             & Medicina      \\
        diabetes         & 768            & 8             & 2             & Medicina      \\
        yeast            & 1483           & 8             & 10            & Biología      \\
        ecoli            & 336            & 7             & 8             & Biología      \\
        iris             & 149            & 4             & 3             & Biología      \\ \hline
      \end{tabular}
      \caption{Información de conjuntos de datos por número de características}
      \label{tab:datasets_info}
    \end{table}
  \end{columns}
\end{frame}

\begin{frame}
  \frametitle{Función fitness}
  \begin{enumerate}
    \item La función \textit{fitness} se construye con:
          \begin{itemize}
            \item \textbf{Accuracy} al $90\%$
            \item \textbf{Reducción} al $10\%$
          \end{itemize}
    \item Se define como: \begin{equation}
            fitness = acc\cdot\alpha + red\cdot(1-\alpha)
            \label{eq:fitness}
          \end{equation}
          Donde $\alpha$ es la ponderación dada a la precisión o \textit{accuracy}
    \item Para el cálculo de precisión se usan los clasificadores \textbf{kNN} y \textbf{SVC}.           
  \end{enumerate}
\end{frame}


\begin{frame}
  \frametitle{Parámetros de los algoritmos}
  \begin{table}[htp]
    \centering
    \begin{tabular}{c|c}
      \hline
      \textbf{Algoritmo} & \textbf{Parámetros}                                                                                                                          \\
      \hline
      GOA                & \begin{tabular}[c]{@{}c@{}}$c_{min}$: 0.00001\ $c_{max}$: 1\ F: 0.5\ L: 1.5\end{tabular}         \\
      WOA                & Parámetro espiral: 1                                                                                                                         \\
      ABCO               & \begin{tabular}[c]{@{}c@{}}Abeja empleada: 3\ Abeja vigilante: 3\ Límite: 3\end{tabular}                                                     \\
      BA                 & \begin{tabular}[c]{@{}c@{}}$\alpha$: 0.9\ $\gamma$: 0.9\ $f_{min}$: 0\ $f_{max}$: 2\end{tabular} \\
      PSO                & \begin{tabular}[c]{@{}c@{}}w: 0.9\ $c_1$: 2\ $c_2$: 2\end{tabular}                                                                           \\
      FA                 & \begin{tabular}[c]{@{}c@{}}$\alpha_0$: 0.5\ $\beta_0$: 0.2\ $\gamma_0$: 1\end{tabular}                                                       \\
      GA                 & \begin{tabular}[c]{@{}c@{}}Ratio de cruce: 1\ Ratio de mutación: 0.05\ Elite: 2\ $\eta$: 1\ $\alpha$: $\sqrt{0.3}$\end{tabular}  \\
      ACO                & \begin{tabular}[c]{@{}c@{}}$\alpha$: 1\ Q: 1\ Feromona inicial: 0.1\ Ratio de evaporación: 0.049\end{tabular}                                \\
      CS                 & \begin{tabular}[c]{@{}c@{}}Ratio de descubrimiento: 0.25\ $\alpha$: 1\ $\lambda$: 1.5\end{tabular}                                           \\
      DE                 & \begin{tabular}[c]{@{}c@{}}F: 0.5\ Cr: 0.1\end{tabular}                                                                                      \\
      \hline
    \end{tabular}
    \caption{Parámetros de diferentes algoritmos de optimización}
  \end{table}
\end{frame}