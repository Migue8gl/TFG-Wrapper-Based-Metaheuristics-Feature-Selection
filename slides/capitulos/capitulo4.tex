\section{Diseño Experimental}
\subsection{Conjuntos de datos}
\begin{frame}
  \frametitle{Conjuntos de datos}
  \begin{columns}
    \column{0.5\textwidth}
    \begin{enumerate}
      \item Se escogen conjuntos de datos por:
            \begin{itemize}
              \item \textbf{Variedad de áreas}
              \item \textbf{Diversidad de problemas}
              \item \textbf{Número de características}
              \item \textbf{Relevancia práctica}
            \end{itemize}
    \end{enumerate}
    \column{0.5\textwidth}
    \begin{table}[htp]
      \centering
      \tiny
      \setlength{\tabcolsep}{4pt}
      \begin{tabular}{ l r r r l }
        \hline
        \textbf{Dataset} & \textbf{Inst.} & \textbf{Car.} & \textbf{Cls.} & \textbf{Área} \\ \hline
        sonar            & 207            & 60            & 2             & Biología      \\
        spambase-460     & 459            & 54            & 2             & Informática   \\
        spectf-heart     & 348            & 44            & 2             & Medicina      \\
        waveform5000     & 5000           & 40            & 3             & Física        \\
        ionosphere       & 350            & 34            & 2             & Meteorología  \\
        dermatology      & 366            & 34            & 6             & Medicina      \\
        wdbc             & 568            & 29            & 2             & Medicina      \\
        parkinsons       & 200            & 22            & 2             & Medicina      \\
        zoo              & 101            & 18            & 7             & Biología      \\
        wine             & 182            & 13            & 3             & Alimentación  \\
        breast-cancer    & 286            & 9             & 2             & Medicina      \\
        diabetes         & 768            & 8             & 2             & Medicina      \\
        yeast            & 1483           & 8             & 10            & Biología      \\
        ecoli            & 336            & 7             & 8             & Biología      \\
        iris             & 149            & 4             & 3             & Biología      \\ \hline
      \end{tabular}
      \caption{Información de conjuntos de datos por número de características}
      \label{tab:datasets_info}
    \end{table}
  \end{columns}
\end{frame}

\begin{frame}
  \frametitle{Diseño Experimental}
  \begin{enumerate}
    \item La función \textit{fitness} se construye con:
          \begin{itemize}
            \item \textbf{Accuracy} al $90\%$.
            \item \textbf{Reducción} al $10\%$.
          \end{itemize}
    \item Se define como: \begin{equation}
            fitness = acc\cdot\alpha + red\cdot(1-\alpha)
            \label{eq:fitness}
          \end{equation}
          Donde $\alpha$ es la ponderación dada a la precisión o \textit{accuracy}.
  \end{enumerate}
\end{frame}