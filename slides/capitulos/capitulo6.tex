\section{Conclusiones}

\begin{frame}
\frametitle{¿Algoritmos específicos vs. originales?}
\begin{itemize}
\item<1-> \textcolor{green}{Ambos enfoques son capaces de reducir características}
\item<2-> \textcolor{green}{Los algoritmos binarios son mucho más eficaces}
\end{itemize}
\end{frame}

\begin{frame}
\frametitle{¿Algoritmos recientes vs. clásicos?}
\begin{itemize}
\item<1-> \textcolor{blue}{Algunos recientes (GWO, CS) muestran excelente rendimiento}
\item<2-> \textcolor{blue}{Algoritmos clásicos (PSO, GA) siguen siendo competitivos}
\end{itemize}
\end{frame}

\begin{frame}
\frametitle{¿Cuáles son los algoritmos más prometedores?}
\begin{itemize}
\item<1-> \textcolor{orange}{GWO y CS destacan en versiones continuas y binarias}
\end{itemize}
\end{frame}

\begin{frame}
\frametitle{¿Eficacia binaria vs. continua?}
\begin{itemize}
\item<1-> \textcolor{purple}{Generalmente similar}
\item<2-> \textcolor{purple}{Ligeras ventajas para versiones continuas en algunos casos y viceversa}
\end{itemize}
\end{frame}

\begin{frame}
\frametitle{¿Qué opciones elegir según el contexto?}
\begin{itemize}
\item<1-> \textcolor{red}{Mejor fitness general:}
    \begin{itemize}
    \item Continuo: CS, GWO
    \item Binario: bGWO, bPSO
    \end{itemize}
\item<2-> \textcolor{red}{Mejor reducción de características:}
    \begin{itemize}
    \item Continuo: CS, GWO
    \item Binario: ACO
    \end{itemize}
\item<3-> \textcolor{red}{Mayor eficiencia temporal:}
    \begin{itemize}
    \item FA (tanto en versión continua como binaria)
    \end{itemize}
\end{itemize}
\end{frame}